\documentclass{amsart}

\usepackage{mathrsfs}
\usepackage{tikz-cd}
\usepackage{amssymb}

\theoremstyle{definition}
\newtheorem{exercise}{Exercise}

\newcommand{\C}{\mathbb{C}}
\newcommand{\Q}{\mathbb{Q}}
\DeclareMathOperator{\GL}{GL}
\DeclareMathOperator{\SO}{SO}
\DeclareMathOperator{\UU}{U}
\DeclareMathOperator{\OO}{O}
\DeclareMathOperator{\Sp}{Sp}
\DeclareMathOperator{\HH}{H}
\DeclareMathOperator{\Td}{Td}
\DeclareMathOperator{\Pic}{Pic}
\DeclareMathOperator{\ch}{ch}


\title{Exercises for 520-2}

\begin{document}

\maketitle
\thispagestyle{empty}

\section{March 28, 2017}

\begin{exercise}
    Let $E\to X$ be a smooth complex vector bundle and $\mathscr{E}$ be its sheaf of smooth sections.
    Show that if $\mathscr{U}=\{U_\alpha\}_{\alpha\in I}$ is a good cover for $X$ then $\check H^k(\mathscr{U},\mathscr{E})=0$
    for $k>0$.
    [Hint: fix a partition of unity $\eta$ for $\mathscr{U}$ and define a contracting homotopy
        $s:\check C^k(\mathscr{U},\mathscr{E})\to \check C^{k-1}(\mathscr{U},\mathscr{E})$ by
        \begin{equation}
            (s\omega)_{\alpha_0\cdots\alpha_{k-1}} = \sum_\alpha \eta_\alpha \omega_{\alpha_0\cdots \alpha_{k-1}\alpha}.
        \end{equation}
        where $\omega\in\check C^k(\mathscr{U},\mathscr{E})$.]
\end{exercise}

\begin{exercise}
    Construct explicitly the long exact sequence for \v Cech cohomology.
\end{exercise}

\begin{exercise}
    Check that for a good cover $\mathscr{U}$ of $X$, the group $\check \HH^1(\mathscr{U},\underline{\GL(1,\C)})$
    is isomorphic to the group $\Pic X$ of isomorphism classes of complex line bundles on $X$.
\end{exercise}

\begin{exercise}
    Recall that the Todd genus of a complex line bundle $L\to X$ is the (inhomogeneous) cohomology
    class $\Td(L)=f(c_1(L))$ where
    $c_1(L)$ is the first Chern class of $L$ and $f(x)=x/(1-e^{-x})$. Compute the Todd genus of
    a rank 2 complex vector bundle $E\to X$ in terms of $c_1(E)$ and $c_2(E)$.
\end{exercise}

\section{March 30, 2017}

\begin{exercise}
    Recall that the Chern character of $E=L_1\oplus \cdots \oplus L_r$ is defined
    \begin{equation*}
        \ch(E) = \sum_{i=1}^r \ch(L_i) = \sum_{i=1}^r  \exp c_1(L_i) = \sum_{k=0}^\infty\ch_k(E)
    \end{equation*}
    where $\ch_k(E)\in\HH^{2k}(X;\Q)$.
    Check that $\ch_0(E)=r$ and $\ch_1(E)=c_1(E)$, then compute $\ch_2$ and $\ch_3$ in terms
    of $c_i$.
\end{exercise}

\begin{exercise}
    Let $X$ be a compact complex manifold and $E\to X$ be a holomorphic vector bundle whose
    sheaf of holomorphic sections we write $\mathscr{E}$. The Hirzebruch-Riemann-Roch theorem is the equality
    \begin{equation*}
        \sum_{k=0}^\infty (-1)^k\dim \HH^k(X, \mathscr{E}) =: \chi(E) = \int_X \ch(E)\Td(X)
    \end{equation*}
    where $\Td(X):=\Td(TX)$. We will prove this theorem later in the course.
    First recover the Riemann-Roch theorem from this equality, then
    prove the theorem for $X=\C P^n$ and $E=\C P^n\times\C$ the trivial bundle.
\end{exercise}

\end{document}
